\documentclass{article}
\usepackage{hyperref}
\usepackage[utf8]{inputenc}

\def\UrlBreaks{\do\/\do-}

\newcommand{\zarya}{the game engine}
\newcommand{\Zarya}{The game engine}

\title{
	Zarya: A Game Engine \\
	\large{TDA572 Game engine architecture}
}
\author{Mårten Åsberg}
\date{March 2022}

\begin{document}

\maketitle

In this report I will be discussing the various features of my game engine.
Since this is written completely from scratch I will not give implementation descriptions of each subsystem, but rather a overview of the whole engine and a deeper description of some of the more interesting features.

\Zarya{} got its name from the first module of the International Space Station\cite{NASA99}, as it really only contains the most basic module needed for a game engine.
The goal is of \zarya{} to be easily extended with more modules that provide more features.
All the while keeping the required modules as small, and optional, as possible.

\section*{Dependency Injection}

The basis for the modular design of \zarya{} is dependency injection, or DI.
DI is often defined as a set of software design principles and patterns that helps writing loosely coupled code, which often results in a more maintainable code base \cite{DI18}.

Most of the implementation of the DI system is provided through a library\footnote{\href{https://www.nuget.org/packages/Microsoft.Extensions.DependencyInjection}{Microsoft.Extensions.DependencyInjection on NuGet Gallery}}.
It is kicked-off by \zarya{} in \texttt{GameBuilder} where a \texttt{IServiceCollection} object handles the different services.
The \texttt{GameBuilder} class is however only use during start-up, when the game is up and running it is the responsibility of a supplied \texttt{IGameManager} implementation to instantiate objects that depend on the services.

The built-in implementation of \texttt{IGameManager}, \texttt{SilkWindow}, show how the DI concept of lifetime is used to promote loose coupling between related objects.
The lifetime of services such as \texttt{Transform2D} and \texttt{PhysicsBody2D} is ``scoped'', which in general DI means that they will be created once for each scope and will only be destroyed when that scope is destroyed.
In \zarya{} scopes are centered around objects, for example when creating a ``bullet'' object that requires a \texttt{Transform2D} and a \texttt{PhysicsBody2D} both the ``bullet'' and the \texttt{PhysicsBody2D} receives the same \texttt{Transform2D} instance without either knowing of the others need for it.
The other lifetime used in \zarya{} is singletons. This life time is used for things like the \texttt{IGameManager} and \texttt{PhysicsManager2D}. These are services that there only ever will exist one instance of.

\section*{Subsystems}

Even though \zarya{} rarely require you to use, or even include, any specific subsystems it come bundled with a few default ones that can be useful in most situations.
In fact, the only subsystem that is required, if it even can be called a \emph{sub}system, is the \texttt{IGameManager}.
And as you probably can tell, there is no specific implementation of this subsystem that is required.

This section will detail the major included subsystem that comes with \zarya{}.

\subsection*{Silk.NET Game Manager}

Silk.NET\footnote{\href{https://www.nuget.org/packages/Silk.NET}{Silk.NET on NuGet Gallery}} is a performant .NET graphics library with high-level utilities for platform-agnostic windowing and input.
Basically, it is a modern incarnation of SDL for C\#.

\sloppy This makes Silk.NET the perfect basis for a combined \texttt{IGameManager}, \texttt{IInputManager}, and rendering system.
When instantiating a Silk.NET window we can listen to events for \texttt{Load} and \texttt{Update}, these fit perfectly with \texttt{IGameManager}s \texttt{Initialize} and \texttt{Update} events.
When a window has loaded we can also listen to various input events, which can be used to implement the events of \texttt{IInputManager}.

Silk.NET windows also gives us access to lower-level bindings to a OpenGl context for rendering.
These bindings are far too low-level to be doing anything useful with directly, but with other services they can easily be molded into something more useful.

\subsection*{SkiaSharp Rendering}

SkiaSharp\footnote{\href{https://www.nuget.org/packages/SkiaSharp}{SkiaSharp on NuGet Gallery}} is a C\# wrapper for Googles Skia Graphics Engine that provides easy to use 2D graphics.
By giving the OpenGL context from a Silk.NET window to SkiaSharp we suddenly find ourselves with a useful 2D rendering system.

In \zarya{}, anything that implements \texttt{ISkiaSharpRenderable} can be rendered to screen using SkiaSharp.
This subsystem comes with two built-in classes that can be rendered, but using this interface the limits to rendering is just under the limits of SkiaSharp.

\texttt{SkiaSharpSprite} can be used to render an image. Simply provide the file path to an image and the sprite class will take care of the reset.

During debugging it can be nice to see colliders of your physics object visually. That is where \texttt{DebugCircleCollider} comes in, it is a combined circle collider and SkiaSharp renderable. Pretty handy.

\subsection*{The Physics System}

Many games needs a physics system, even if they only need something as ``simple'' as detecting intersection checks \cite{MH22}.
For that reason \zarya{} comes with a built-in physics subsystem.

The physics system is mostly a copy of the physics system in Shard\footnote{Shard is the game engine provided by Michael Heron for the course \href{https://student.portal.chalmers.se/en/chalmersstudies/courseinformation/Pages/SearchCourse.aspx?course_id=24182&parsergrp=3}{TDA572 -- Game engine architecture}}, but adapted to work in the framework of \zarya{}.
However it is also scaled down, this version only supports circle colliders and not rectangular colliders as the original does.
This is because the goal was to replace it with a wrapper around Box2D\footnote{\href{https://box2d.org/}{Box2D} is a 2D physics engine}, but that never came to be.

\subsection*{Input Binding}

This is the only subsystem that is truly original.
The others were more (the physics system) or less (Silk.NET and SkiaSharp rendering) reimplementations of what Shard did.
The input binding subsystem provides an easy to use system for binding input events from keyboards, mice, and gamepads to statically represented points in code.

To achieve this, a developer can extend the \texttt{InputBase} record and annotate properties with one, or more, of the \texttt{*InputAttribute} attributes.
At startup these attributes will be used to dynamically construct method bodies that match their statically described types and provide the functionality described in the attributes.

This system comes with a few benefits over other input binding systems.
For example, mapping strings to input events gives no compile time checks for type or usage correctness.

This subsystem is implemented in the \texttt{Zarya.Input} namespace, and usage example of it can be found in the \texttt{GameTest.Input} record.

\section*{Grading Expectations}

As discussed with the examiner, I expect my efforts to definitely award me a passing grade.
I would give myself a grade of 4, not because I don't think I deserve a 5 but because I'm too Swedish and follows the law of Jante.

\newpage

\begin{thebibliography}{2}
	\bibitem{NASA99}
		NASA, ``NASA Facts. The Service Module: A Cornerstone of Russian International Space Station Modules,'' 1999.
		[Online]. Available: \url{https://web.archive.org/web/20210117004625/https://spaceflight.  nasa.gov/spacenews/factsheets/pdfs/servmod.pdf} (accessed on: 2022-03-09).
	\bibitem{DI18}
		M. Seemann, \emph{Dependency Injection: Principles, Practices, and Patterns}.
		Shelter Island, USA: Manning Publications, 2018.
	\bibitem{MH22}
		M. Heron, ``Lecture Series'' in \emph{TDA572 -- Game engine architecture}, Gothenburg, Sweden, 2022.
\end{thebibliography}

\end{document}
